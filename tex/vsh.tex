%
%% Beginning of file 'sample62.tex'
%%
%% Modified 2018 January
%%
%% This is a sample manuscript marked up using the
%% AASTeX v6.2 LaTeX 2e macros.
%%
%% AASTeX is now based on Alexey Vikhlinin's emulateapj.cls 
%% (Copyright 2000-2015).  See the classfile for details.

%% AASTeX requires revtex4-1.cls (http://publish.aps.org/revtex4/) and
%% other external packages (latexsym, graphicx, amssymb, longtable, and epsf).
%% All of these external packages should already be present in the modern TeX 
%% distributions.  If not they can also be obtained at www.ctan.org.

%% The first piece of markup in an AASTeX v6.x document is the \documentclass
%% command. LaTeX will ignore any data that comes before this command. The 
%% documentclass can take an optional argument to modify the output style.
%% The command below calls the preprint style  which will produce a tightly 
%% typeset, one-column, single-spaced document.  It is the default and thus
%% does not need to be explicitly stated.
%%
%%
%% using aastex version 6.2
\documentclass[times]{aastex631}
\usepackage{amsmath}
\usepackage{natbib}

%% The default is a single spaced, 10 point font, single spaced article.
%% There are 5 other style options available via an optional argument. They
%% can be envoked like this:
%%
%% \documentclass[argument]{aastex62}
%% 
%% where the layout options are:
%%
%%  twocolumn   : two text columns, 10 point font, single spaced article.
%%                This is the most compact and represent the final published
%%                derived PDF copy of the accepted manuscript from the publisher
%%  manuscript  : one text column, 12 point font, double spaced article.
%%  preprint    : one text column, 12 point font, single spaced article.  
%%  preprint2   : two text columns, 12 point font, single spaced article.
%%  modern      : a stylish, single text column, 12 point font, article with
%% 		  wider left and right margins. This uses the Daniel
%% 		  Foreman-Mackey and David Hogg design.
%%  RNAAS       : Preferred style for Research Notes which are by design 
%%                lacking an abstract and brief. DO NOT use \begin{abstract}
%%                and \end{abstract} with this style.
%%
%% Note that you can submit to the AAS Journals in any of these 6 styles.
%%
%% There are other optional arguments one can envoke to allow other stylistic
%% actions. The available options are:
%%
%%  astrosymb    : Loads Astrosymb font and define \astrocommands. 
%%  tighten      : Makes baselineskip slightly smaller, only works with 
%%                 the twocolumn substyle.
%%  times        : uses times font instead of the default
%%  linenumbers  : turn on lineno package.
%%  trackchanges : required to see the revision mark up and print its output
%%  longauthor   : Do not use the more compressed footnote style (default) for 
%%                 the author/collaboration/affiliations. Instead print all
%%                 affiliation information after each name. Creates a much
%%                 long author list but may be desirable for short author papers
%%
%% these can be used in any combination, e.g.
%%
%% %documentclass[twocolumn,linenumbers,trackchanges]{aastex62}
%%
%% AASTeX v6.* now includes \hyperref support. While we have built in specific
%% defaults into the classfile you can manually override them with the
%% \hypersetup command. For example,
%%
%%\hypersetup{linkcolor=red,citecolor=green,filecolor=cyan,urlcolor=magenta}
%%
%% will change the color of the internal links to red, the links to the
%% bibliography to green, the file links to cyan, and the external links to
%% magenta. Additional information on \hyperref options can be found here:
%% https://www.tug.org/applications/hyperref/manual.html#x1-40003
%%
%% If you want to create your own macros, you can do so
%% using \newcommand. Your macros should appear before
%% the \begin{document} command.
%%
\newcommand{\vdag}{(v)^\dagger}
\newcommand\aastex{AAS\TeX}
\newcommand\latex{La\TeX}

\newcommand{\nab}{\vec{\nabla}}
\newcommand{\nabang}{\vec{\nabla}_{\omega}}
\newcommand{\curl}{\vec{\nabla}\times}
\newcommand{\bpol}{\vec{B}_{\mathrm{pol}}}
\newcommand{\btor}{\vec{B}_{\mathrm{tor}}}
\newcommand{\nin}{\noindent}
\newcommand{\ptheta}{\partial_{\theta}}
\newcommand{\pphi}{\partial_{\phi}}
\newcommand{\dOmega}{\mathrm{d}\Omega}
\newcommand{\eqn}[1]{Eq.\,(#1)}
\newcommand{\ypvsh}[2]{\mathbf{Y}_{#1}^{(#2)}}
\newcommand{\yvsh}[2]{\mathbf{Y}_{#1}^{#2}}
\allowdisplaybreaks

% below not used yet, can't decide which notation is best
%\newcommand{\bfy}[2]{\vec{\mathbf{Y}}_{#1}^{#2}}
%\newcommand{\lampar}{\left(\lambda\right)}


%% Tells LaTeX to search for image files in the 
%% current directory as well as in the figures/ folder.
\graphicspath{{./}{figures/}}

%% Reintroduced the \received and \accepted commands from AASTeX v5.2

\received{\today}
%\revised{January 7, 2018}
%\accepted{\today}

%% Command to document which AAS Journal the manuscript was submitted to.
%% Adds "Submitted to " the arguement.

%\submitjournal{ApJ}

%% Mark up commands to limit the number of authors on the front page.
%% Note that in AASTeX v6.2 a \collaboration call (see below) counts as
%% an author in this case.
%
%\AuthorCollaborationLimit=3
%
%% Will only show Schwarz, Muench and "the AAS Journals Data Scientist 
%% collaboration" on the front page of this example manuscript.
%%
%% Note that all of the author will be shown in the published article.
%% This feature is meant to be used prior to acceptance to make the
%% front end of a long author article more manageable. Please do not use
%% this functionality for manuscripts with less than 20 authors. Conversely,
%% please do use this when the number of authors exceeds 40.
%%
%% Use \allauthors at the manuscript end to show the full author list.
%% This command should only be used with \AuthorCollaborationLimit is used.

%% The following command can be used to set the latex table counters.  It
%% is needed in this document because it uses a mix of latex tabular and
%% AASTeX deluxetables.  In general it should not be needed.
%\setcounter{table}{1}

%%%%%%%%%%%%%%%%%%%%%%%%%%%%%%%%%%%%%%%%%%%%%%%%%%%%%%%%%%%%%%%%%%%%%%%%%%%%%%%%
%%
%% The following section outlines numerous optional output that
%% can be displayed in the front matter or as running meta-data.
%%
%% If you wish, you may supply running head information, although
%% this information may be modified by the editorial offices.
\shorttitle{Vector Spherical Harmonics in Fortran}
\shortauthors{J.G. Elfritz}
%%
%% You can add a light gray and diagonal water-mark to the first page 
%% with this command:
% \watermark{text}
%% where "text", e.g. DRAFT, is the text to appear.  If the text is 
%% long you can control the water-mark size with:
%  \setwatermarkfontsize{dimension}
%% where dimension is any recognized LaTeX dimension, e.g. pt, in, etc.
%%
%%%%%%%%%%%%%%%%%%%%%%%%%%%%%%%%%%%%%%%%%%%%%%%%%%%%%%%%%%%%%%%%%%%%%%%%%%%%%%%%

%% This is the end of the preamble.  Indicate the beginning of the
%% manuscript itself with \begin{document}.

\begin{document}

%\title[Proofs of magnetofrictional integrals]{PROOFS FOR ANGULAR COUPLINGS IN MAGNETOFRICTIONAL DYNAMICS}

%\title{EXACT SPECTRAL INTEGRALS FOR INDUCTION FIELDS OF ARBITRARY ORDER IN POLOIDAL-TOROIDAL REPRESENTATION}

\title{A Fortran Module For Vector Spherical Harmonics Computations}

%% LaTeX will automatically break titles if they run longer than
%% one line. However, you may use \\ to force a line break if
%% you desire. In v6.2 you can include a footnote in the title.

%% A significant change from earlier AASTEX versions is in the structure for 
%% calling author and affilations. The change was necessary to implement 
%% autoindexing of affilations which prior was a manual process that could 
%% easily be tedious in large author manuscripts.
%%
%% The \author command is the same as before except it now takes an optional
%% arguement which is the 16 digit ORCID. The syntax is:
%% \author[xxxx-xxxx-xxxx-xxxx]{Author Name}
%%
%% This will hyperlink the author name to the author's ORCID page. Note that
%% during compilation, LaTeX will do some limited checking of the format of
%% the ID to make sure it is valid.
%%
%% Use \affiliation for affiliation information. The old \affil is now aliased
%% to \affiliation. AASTeX v6.2 will automatically index these in the header.
%% When a duplicate is found its index will be the same as its previous entry.
%%
%% Note that \altaffilmark and \altaffiltext have been removed and thus 
%% can not be used to document secondary affiliations. If they are used latex
%% will issue a specific error message and quit. Please use multiple 
%% \affiliation calls for to document more than one affiliation.
%%
%% The new \altaffiliation can be used to indicate some secondary information
%% such as fellowships. This command produces a non-numeric footnote that is
%% set away from the numeric \affiliation footnotes.  NOTE that if an
%% \altaffiliation command is used it must come BEFORE the \affiliation call,
%% right after the \author command, in order to place the footnotes in
%% the proper location.
%%
%% Use \email to set provide email addresses. Each \email will appear on its
%% own line so you can put multiple email address in one \email call. A new
%% \correspondingauthor command is available in V6.2 to identify the
%% corresponding author of the manuscript. It is the author's responsibility
%% to make sure this name is also in the author list.
%%
%% While authors can be grouped inside the same \author and \affiliation
%% commands it is better to have a single author for each. This allows for
%% one to exploit all the new benefits and should make book-keeping easier.
%%
%% If done correctly the peer review system will be able to
%% automatically put the author and affiliation information from the manuscript
%% and save the corresponding author the trouble of entering it by hand.

\correspondingauthor{Justin G. Elfritz}
\email{justin.elfritz@gmail.com}

\author{Justin G. Elfritz}
%\affil{EOXS Scientific, Cumberland, MD 21502, United States}
\affil{IBM, IBM Consulting, AI \& Analytics Practice, Poughkeepsie NY, United States of America}

%% Note that the \and command from previous versions of AASTeX is now
%% depreciated in this version as it is no longer necessary. AASTeX 
%% automatically takes care of all commas and "and"s between authors names.

%% AASTeX 6.2 has the new \collaboration and \nocollaboration commands to
%% provide the collaboration status of a group of authors. These commands 
%% can be used either before or after the list of corresponding authors. The
%% argument for \collaboration is the collaboration identifier. Authors are
%% encouraged to surround collaboration identifiers with ()s. The 
%% \nocollaboration command takes no argument and exists to indicate that
%% the nearby authors are not part of surrounding collaborations.

%% Mark off the abstract in the ``abstract'' environment. 
\begin{abstract}

The \textit{Tensor Spherical Harmonics} (TSH) are crucial ingredients to robust representation of curvilinear coordinates and reference frames in mathematical physics. A special limit of the TSH called the \textit{Vector Spherical Harmonics} (VSH) are the focus of this manuscript. Fast and efficient calculation of the VSH is required for many high-performance computing applications, including for magneto-hydrodynamics (MHD), quantum mechanical systems, and other spectral codes. We present ...

\end{abstract}

\tableofcontents

%% Keywords should appear after the \end{abstract} command. 
%% See the online documentation for the full list of available subject
%% keywords and the rules for their use.

%\keywords{, notices --- 
%miscellaneous --- catalogs --- surveys}

%% From the front matter, we move on to the body of the paper.
%% Sections are demarcated by \section and \subsection, respectively.
%% Observe the use of the LaTeX \label
%% command after the \subsection to give a symbolic KEY to the
%% subsection for cross-referencing in a \ref command.
%% You can use LaTeX's \ref and \label commands to keep track of
%% cross-references to sections, equations, tables, and figures.
%% That way, if you change the order of any elements, LaTeX will
%% automatically renumber them.
%%
%% We recommend that authors also use the natbib \citep
%% and \citet commands to identify citations.  The citations are
%% tied to the reference list via symbolic KEYs. The KEY corresponds
%% to the KEY in the \bibitem in the reference list below. 



\section{Background}

The tensor spherical harmonics (TSH) are irreducible tensor products of scalar spherical harmonics $Y_L^M(\Omega)$ and spin functions $\chi_S$ \citep{Edmonds1960}. Throughout this text we adopt the normalization for $Y_l^m$ as follows:

\begin{equation}
  Y_l^m = \sqrt\frac{2l+1}{4\pi}\sqrt{\frac{\left(l-m\right)!}{\left(l+m\right)!}}P_l^m(\cos\theta)e^{im\phi}
\end{equation}

\noindent An arbitrary TSH may be written as a sum over non-zero eigenstates and projections as

\begin{equation}
  \mathbf{Y}_{J M}^{L S} = \sum_{m, \sigma} C_{L m S \sigma}^{J M} Y_{L}^{M}\chi_{S_{\sigma}},
\end{equation}
  
\nin where the non-negative integer or half-integer $S$ indicates the tensor rank, and $C_{a\alpha b \beta}^{c\gamma}$ is a Clebsch-Gordan coefficient. Applications in physics commonly require the evaluation of integrals of products of scalar (\textit{i.e.}, $S=0$) spherical harmonics, and their angular spatial derivatives, on the 3D unit sphere \citep{Radler73,Winch74,Barrera1985}. Such a problem may be cast in terms of the $S=1$ vector spherical harmonics with the generalised form

%\begin{equation}\label{eq:initintegral}
%  S(n_1,m_1,n_2,...n_{N_i -1},m_{N_i -1};\lambda_1,\lambda_2,...) = \int_{4\pi}\dOmega\,\left(\vec{\mathbf{Y}}_{n_1 m_1}^{\left(\lambda_1\right)}\cdot\vec{\mathbf{Y}}_{n_2 m_2}^{\left(\lambda_2\right)}\right)\cdot ...\cdot\left(\vec{\mathbf{Y}}_{n_{N_i-1}m_{N_i-1}}^{\left(\lambda_{N_i-1}\right)}\cdot\vec{\mathbf{Y}}_{n_{N_i} m_{N_i}}^{\left(\lambda_{N_i}\right)*}\right).
%\end{equation}

\begin{equation}\label{eq:initintegral}
  S(\vec{n},\vec{m};\vec{\lambda}) = \int_{4\pi}\dOmega\,\left[\left(\yvsh{n_1 m_1}{\left(\lambda_1\right)}\times\yvsh{n_2 m_2}{\left(\lambda_2\right)}\right)\times\yvsh{n_3 m_3}{\left(\lambda_3\right)}\times...\times\yvsh{n_{N}m_{N}}{\left(\lambda_{N}\right)}\right]\cdot \yvsh{n' m'}{\left(\lambda'\right)*}.
\end{equation}

\nin  In Eq.\,(\ref{eq:initintegral}) components of the vector spherical harmonics (VSH) of order $n$ and azimuthal degree $m$, written as $\yvsh{n m}{L}$, are encoded into the polar-VSH (pVSH) via the orientation index $\lambda$. These pVSH are denoted by $\ypvsh{n m}{\lambda}$, following the notation of \cite{AkheizerBook} and \cite{QTAM} which will be used henceforth. Here the VSH are constructed in the spherical basis, such that the unit vectors $\chi_{1\mu}=\hat{e}_{\mu}$ correspond to the longitudinal direction ($\mu=\lambda=-1$) and the transverse directions ($\mu=0,+1$; $\lambda=0,+1$) relative to the radial unit normal $\hat{r}=\vec{r}/r$. These conventions are chosen for situations where it is beneficial to decompose the system into poloidal and toroidal components \citep{KrauseRadlerBook}, . The VSH and pVSH components are related to the scalar spherical harmonics $Y_n^m$ and their spatial derivatives (in polar coordinates) by

\begin{align}
\hat{r}Y_n^m = & \ypvsh{n m}{-1} \; = \frac{1}{\sqrt{2n+1}}\left[\sqrt{n}\yvsh{n m}{n-1} - \sqrt{n+1}\yvsh{n m}{n+1}\right]\label{eq:lam-1}\\
\frac{1}{\sqrt{\Lambda_n}}\widehat{\textbf{L}}Y_n^m = -\frac{i}{\sqrt{\Lambda_n}}\hat{r}\times\nabang Y_n^m = & \;\ypvsh{n m}{0}\;\, = \yvsh{n m}{n}\label{eq:lam0}\\
\frac{1}{\sqrt{\Lambda_n}}\nabang Y_n^m = & \;\ypvsh{n m}{+1} = \frac{1}{\sqrt{2n+1}}\left[\sqrt{n+1}\yvsh{n m}{n-1} + \sqrt{n}\yvsh{n m}{n+1}\right].\label{eq:lam+1}
\end{align}

\nin The VSH and pVSH independently constitute complete orthonormal bases, and are therefore candidates for describing vector fields in poloidal-toroidal decomposition. Here $\sqrt{\Lambda_n}=\sqrt{n(n+1)}$ is the eigenvalue for the angular momentum operator $\widehat{\textbf{L}}$.

\section{Identities involing Vector Spherical Harmonics}

\begin{align}
\curl\left(f(r)\ypvsh{nm}{-1}\right) = & -\frac{i}{R_n}f(r)\ypvsh{nm}{0} \\
\curl\left(f(r)\ypvsh{nm}{+1}\right) = & \;\frac{i}{r}\frac{d}{dr}\left(rf(r)\right)\ypvsh{nm}{0} \\
\curl\left(f(r)\ypvsh{nm}{0}\right) = & \;\frac{i}{R_n}\left[f(r)\ypvsh{nm}{-1}+\frac{d}{dr}\left(R_nf(r)\right)\ypvsh{nm}{+1}\right]
\end{align}

\begin{align}
  \ypvsh{kl}{+1}\cdot\ypvsh{nm}{+1} = & \frac{1}{\sqrt{\Lambda_k\Lambda_n}}\nabang Y_k^l\cdot\nabang Y_n^m\nonumber \\
  = & \sum_{L}(-1)^{k+n+L-1}\frac{(2k+1)(2n+1)}{\sqrt{4\pi(2L+1)}}\begin{Bmatrix} k & n & L \\ n & k & 1 \end{Bmatrix} C_{k 0 n 0}^{L 0}C_{k l n m}^{L l+m}Y_L^{l+m}\nonumber \\ = & \sum_{L}\frac{2(2k+1)(2n+1)}{\sqrt{4\pi(2L+1)}}\sqrt{\frac{(2k-1)!(2n-1)!}{(2k+2)!(2n+2)!}}(\Lambda_k + \Lambda_n - \Lambda_L) C_{k 0 n 0}^{L 0}C_{k l n m}^{L l+m}Y_L^{l+m}\nonumber\\
  = & \sum_L\frac{\Lambda_k + \Lambda_n - \Lambda_L}{2\sqrt{\Lambda_k\Lambda_n}}I_{k l n m}^{L l+m}Y_L^{l+m}\label{eq:L-dot-L} \\
  \nonumber\\
  \ypvsh{kl}{0}\cdot\ypvsh{nm}{0} = & \frac{-1}{\sqrt{\Lambda_k\Lambda_n}}\nabang Y_k^l\cdot\nabang Y_n^m\nonumber \\
  %= & \sum_{L}(-1)^{k+n+L}\frac{(2k+1)(2n+1)}{\sqrt{4\pi(2L+1)}}\begin{Bmatrix} k & n & L \\ n & k & 1 \end{Bmatrix} C_{k 0 n 0}^{L 0}C_{k l n m}^{L l+m}Y_L^{l+m}\nonumber \\ = & -\sum_{L}\frac{2(2k+1)(2n+1)}{\sqrt{4\pi(2L+1)}}\sqrt{\frac{(2k-1)!(2n-1)!}{(2k+2)!(2n+2)!}}(\Lambda_k + \Lambda_n - \Lambda_L) C_{k 0 n 0}^{L 0}C_{k l n m}^{L l+m}Y_L^{l+m}\nonumber\\
  = & -\sum_L\frac{\Lambda_k + \Lambda_n - \Lambda_L}{2\sqrt{\Lambda_k\Lambda_n}}I_{k l n m}^{L l+m}Y_L^{l+m}\label{eq:grad-dot-grad} \\
   \nonumber\\
%\end{align}
%\begin{align}
   \ypvsh{kl}{+1}\cdot\ypvsh{nm}{0}
  = & \frac{-i}{\sqrt{\Lambda_k\Lambda_n}}\nabang Y_k^l \cdot \left(\hat{r}\times\nabang Y_n^m\right)\nonumber \\
  = &\,(2n+1)(-1)^{n+k+L+1}\sum_{L}C_{k l n m}^{L M}Y_L^M \times \nonumber \\
  & \times\left[\sqrt{\frac{(k+1)(2k-1)}{4\pi(2L+1)}}\begin{Bmatrix} k-1 & n & L \\ n & k & 1\end{Bmatrix}C_{k-1 0 n 0}^{L 0} + \sqrt{\frac{k(2k+3)}{4\pi(2L+1)}}\begin{Bmatrix}k+1 & n & L \\ n & k & 1\end{Bmatrix}C_{k+1 0 n 0}^{L 0}\right] \nonumber\\
  = & \sum_L \sqrt{(k+n+L+2)(k-n+L+1)(k+n-L+1)(-k+n+L)}C_{k+1 0 n 0}^{L 0} C_{k l n m}^{L M} Y_L^M \times \nonumber \\
  & \times \frac{1}{2}\sqrt{\frac{2n+1}{4\pi(2L+1)n(n+1)}}\left[\sqrt{\frac{k+1}{k(2k+1)}}+\sqrt{\frac{k}{(2k+1)(k+1)}}\right] \nonumber\\
  = & \frac{-i}{\sqrt{\Lambda_k\Lambda_n}}\sum_L J_{n m k l}^{L M} Y_L^M\label{eq:grad-dot-L}\\
  \nonumber\\
%\end{align}
%\begin{align}
  \ypvsh{kl}{-1}\cdot\ypvsh{nm}{-1} = & Y_k^l Y_n^m \nonumber \\
  = & \frac{1}{\sqrt{(2k+1)(2n+1)}}\sum_L (-1)^{n+k+L+1}\sqrt{\frac{(2k+1)(2n+1)}{4\pi(2L+1)}}C_{k l n m}^{L M} Y_L^M\times\nonumber\\
  & \times\sum_{\nu,\sigma=\pm 1}\sqrt{\left(2k+\sigma+1\right)\left(2n+\nu+1\right)\left(k+\sigma+\frac{1}{2}\right)\left(n+\nu+\frac{1}{2}\right)}\begin{Bmatrix}k+\sigma & n+\nu & L \\ n & k & 1\end{Bmatrix}C_{k+\sigma 0 n+\nu 0}^{L 0}\nonumber\\
   = & \sum_L I_{k l n m}^{L M} Y_L^M\label{eq:r-dot-r}
\end{align}

\nin In many cases the above relations must be modifed when a VSH component is conjugated, \textit{e.g.} for $\nabang Y_k^l\cdot\nabang Y_n^{m*}$. The complex conjugates of the VSH and pVSH are well-known: 

\begin{equation}
\yvsh{J M}{L *} = (-1)^{J + L + M + 1}\yvsh{J -M}{L}
\end{equation}

\begin{equation}
  \yvsh{J M}{(\lambda)*} = (-1)^{\lambda + M + 1}\ypvsh{J -M}{\lambda}.
\end{equation}

\nin In combination with the VSH scalar products in Eqs.\,(\ref{eq:grad-dot-grad}--\ref{eq:r-dot-r}), it may be directly shown that

\begin{align}
  \ypvsh{kl}{+1}\cdot\yvsh{nm}{(+1)*} = & \frac{1}{\sqrt{\Lambda_k\Lambda_n}}\nabang Y_k^l\cdot\nabang Y_n^{m*} \nonumber\\
  = & \sum_L\frac{\Lambda_k+\Lambda_n-\Lambda_L}{2\sqrt{\Lambda_k\Lambda_n}}I_{k l L m-l}^{n m}Y_L^{m-l *}\\
  \nonumber\\
  \ypvsh{kl}{0}\cdot\yvsh{nm}{(0)*} = & \frac{1}{\sqrt{\Lambda_k\Lambda_n}}\nabang Y_k^l\cdot\nabang Y_n^{m*} \nonumber\\
  = & \sum_L\frac{\Lambda_k+\Lambda_n-\Lambda_L}{2\sqrt{\Lambda_k\Lambda_n}}I_{k l L m-l}^{n m}Y_L^{m-l *}\\
  \nonumber\\
  \ypvsh{kl}{+1}\cdot\yvsh{nm}{(0)*} = & \frac{i}{\sqrt{\Lambda_k\Lambda_n}}\nabang Y_k^l\cdot\left(\hat{r}\times\nabang Y_n^{m*}\right)\nonumber\\
  = & \frac{i}{\sqrt{\Lambda_k\Lambda_n}}\sum_L J_{k l L m-l}^{n m}Y_L^{m-l *} \\
  \ypvsh{nm}{0}\cdot\yvsh{kl}{(+1)*} = & \frac{-i}{\sqrt{\Lambda_k\Lambda_n}}\left(\hat{r}\times\nabang Y_n^{m}\right)\cdot\nabang Y_k^{l*}\nonumber\\
  = & \frac{i}{\sqrt{\Lambda_k\Lambda_n}}\sum_L J_{n m L l-m}^{k l}Y_L^{l-m *} \\
  \ypvsh{kl}{-1}\cdot\yvsh{nm}{(-1)*} = & Y_k^l Y_n^{m*}\nonumber\\
 % = & (-1)^l\sum_L I_{k l n -m}^{L l-m} Y_L^{m-l *} \nonumber\\
  = & \sum_L I_{k l L m-l}^{n m}Y_L^{m-l*}
\end{align}
  

\nin Considering these results, it becomes trivial to verify the quadratic angular couplings presented in \citet{GW1}, which were presented in the context of the magnetic Hall drift in the neutron star crust. 


\section{Notation and identities involving Clebsch-Gordan, 3-j, and 6-j coefficients}\label{s:cgidentities}

\begin{align}
I_{k'\,l'\,k\,l}^{n\,m} = & \sqrt{\frac{(2k'+1)(2k+1)}{4\pi(2n+1)}}C_{k'\,0\,k\,0}^{n\,0}C_{k'\,l'\,k\,l}^{n m}\\
J_{k' l' k l}^{n m} = & -\frac{i}{2}\sqrt{\frac{(2k'+1)(2k+1)}{4\pi(2n+1)}}\sqrt{(k'+k+n+2)(k+n-k')(k'+k-n+1)(k'-k+n+1)}\nonumber\\
 & \times C_{k'+1 0 k 0}^{n 0} C_{k' l' k l}^{n m}
\end{align}

\begin{align}
  \begin{Bmatrix} a & b & a+b \\ d & e & f\end{Bmatrix} = & (-1)^{a+b+d+e} \sqrt{\frac{(2a)!(2b)!(a+b+d+e+1)!(a+b-d+e)!(a+b+d-e)!}{(2a+2b+1)!(-a-b+d+e)!(a+e-f)!(a-e+f)!(a+e+f+1)!}}\times\nonumber\\
    & \;\;\;\;\;\;\times\sqrt{\frac{(-a+e+f)!(-b+d+f)!}{(b+d-f)!(b-d+f)!(b+d+f+1)!}}
    \\
%\end{align}
%\begin{align}
  \begin{Bmatrix} a & a & 1 \\ b & b & f\end{Bmatrix} = & 2(-1)^{a+b+f+1}\sqrt{\frac{(2a-1)!(2b-1)!}{(2a+2)!(2b+2)!}}(a(a+1)+b(b+1)-f(f+1))\\
%\end{align}
%\begin{equation}
  C_{k 0 p-1 0}^{n 0} = & -C_{k 0 p+1 0}^{n 0}\sqrt{\frac{(k+n+p+2)(k+p-n+1)(k+n-p)(-k+n+p+1)}{(k+n+p+1)(k+p-n)(k+n-p+1)(-k+p+n)}}\\
%\end{equation}
%\begin{equation}
  C_{k+1 0 n 0}^{L 0} = & C_{k 0 n+1 0}^{L 0}\sqrt{\frac{(k-n+L)(-k+n+L+1)}{(k-n+L+1)(-k+n+L)}}
%\end{equation}
\end{align}

%%%%%%%%%%%%
\section{Numerical Module}



%\section{Recursion Relations}

%\section{Useful identities}

\bibliography{vsh}

%% Use LaTeX's thebibliography environment to xmark up your reference list.
%% Note \begin{thebibliography} is followed by an empty set of
%% curly braces.  If you forget this, LaTeX will generate the error
%% "Perhaps a missing \item?".
%%
%% thebibliography produces citations in the text using \bibitem-\cite
%% cross-referencing. Each reference is preceded by a
%% \bibitem command that defines in curly braces the KEY that corresponds
%% to the KEY in the \cite commands (see the first section above).
%% Make sure that you provide a unique KEY for every \bibitem or else the
%% paper will not LaTeX. The square brackets should contain
%% the citation text that LaTeX will insert in
%% place of the \cite commands.

%% We have used macros to produce journal name abbreviations.
%% \aastex provides a number of these for the more frequently-cited journals.
%% See the Author Guide for a list of them.

%% Note that the style of the \bibitem labels (in []) is slightly
%% different from previous examples.  The natbib system solves a host
%% of citation expression problems, but it is necessary to clearly
%% delimit the year from the author name used in the citation.
%% See the natbib documentation for more details and options.

%\begin{thebibliography}{}

%\end{thebibliography}

%% This command is needed to show the entire author+affilation list when
%% the collaboration and author truncation commands are used.  It has to
%% go at the end of the manuscript.
%\allauthors

%% Include this line if you are using the \added, \replaced, \deleted
%% commands to see a summary list of all changes at the end of the article.
%\listofchanges

\end{document}

% End of file
